\documentclass{article}
\usepackage{fullpage}
%\usepackage[landscape]{geometry}
\usepackage{amsfonts}
\usepackage{amsmath}
\usepackage{amssymb}
\usepackage{lscape}

\usepackage[margin=0.5in]{geometry}


\begin{document}
\newcommand{\tabincell}[2]{\begin{tabular}{@{}#1@{}}#2\end{tabular}}
\begin{center}
{\Large \bf CS 5130 Homework II}
\vspace*{2em}

Liang Yan   
\\From Computer Science\\
\end{center}


\noindent
% This is the question 1
1. Answer the following questions related to reaching definitions analysis for the CFG
\\
(a) Give the GEN and PRSV sets for each basic block
\\
{\small
\\
$U=\{r1@1,r10@2,r2@3,r11@4,r3@5,r15@6,r4@7,r13@8,r5@9,r14@10,r6@12,r14@13,r4@15,r14@16,\\
r1@17,r7@18,r15@19,r4@21,r11@22,r5@23,r15@24,r6@26,r10@27,r8@28,r11@29\}$
\\
\\
\begin{tabular}{| l | c | l |} 
 \hline
 BB & GEN & PRSV\\
  \hline                       
  B0 & {r1@1,r10@2,r2@3,r11@4,r3@5 } & \tabincell{l}
  {r15@6,r4@7,r13@8,r5@9,r14@10,r6@12,r14@13,r4@15,r14@16,\\
   r7@18,r15@19,r4@21,r5@23,r15@24,r6@26,r10@27,r8@28 } \\
\hline
  B1 & {r15@6,r4@7,r13@8,r5@9,r14@10} & \tabincell{l} {r1@1,r10@2,r2@3,r11@4,r3@5,r6@12,\\
r1@17,r7@18,r11@22,r6@26,r10@27,r8@28,r11@29}\\
\hline
  B2 & {r6@12,r14@13} &
  \tabincell{l}{r1@1,r10@2,r2@3,r11@4,r3@5,r15@6,r4@7,r13@8,r5@9,
    r4@15\\r1@17,r7@18,r15@19,r4@21,r11@22,r5@23,r15@24,r10@27,r8@28,r11@29} \\
\hline
  B3 & {r4@15,r14@16,r1@17,r7@18,r15@19} &
  \tabincell{l}{r10@2,r2@3,r3@5,r13@8,r5@9,r14@10,\\ r6@12,r14@13r5@23,r6@26,r10@27,r8@28} \\
  \hline                       
  B4 & {r4@21,r11@22,r5@23,r15@24} &  \tabincell{l}{r1@1,r10@2,r2@3,r3@5,r13@8,r14@10,r6@12,r14@13,
    r14@16,\\r1@17,r7@18,r6@26,r10@27,r8@28} \\
\hline
  B5 & {r6@26,r10@27,r8@28,r11@29} &
  \tabincell{l}{r1@1,r2@3,r3@5,r15@6,r4@7,r13@8,r5@9,r14@10,r14@13,\\r4@15,r14@16,r1@17,r7@18,
    r15@19,r4@21,r5@23,r15@24} \\
  \hline  
\end{tabular}
\\\\(b)
\\Init.(iter.0)\\
IN(b) = $\emptyset$,
OUT(b)=Gen(b) $\cup$ (IN(b) $\cap$ PRSV(b)) = Gen(b) \\\\
\begin{tabular}{| l | c | l |} 
 \hline
 BB & IN & OUT\\
  \hline                       
  B0 & $\emptyset$ &{r1@1,r10@2,r2@3,r11@4,r3@5 }\\
\hline
  B1 & $\emptyset$ &{r15@6,r4@7,r13@8,r5@9,r14@10} \\
\hline
  B2 & $\emptyset$& {r6@12,r14@13}  \\
\hline
  B3 & $\emptyset$& {r4@15,r14@16,r1@17,r7@18,r15@19}  \\
  \hline                       
  B4 & $\emptyset$& {r4@21,r11@22,r5@23,r15@24} \\
\hline
  B5 & $\emptyset$& {r6@26,r10@27,r8@28,r11@29} \\
  \hline  
\end{tabular}
\\\\iter.1
\\ IN(b) = $\bigcup_{p\in pred(b)} OUT(p) $\\ 
OUT(b)=Gen(b) $\cup$ (IN(b) $\cap$ PRSV(b)) = Gen(b) \\\\
\begin{tabular}{| l | c | l |} 
 \hline
 BB & IN & OUT \\
  \hline                       
  B0 & $\emptyset$ &{r1@1,r10@2,r2@3,r11@4,r3@5 }\\
\hline
  B1 & {r1@1,r10@2,r2@3,r11@4,r3@5,r4@21,r11@22,r5@23,r15@24} & \tabincell{l} {r1@1,r10@2,r2@3,r11@4,r3@5,\\r15@6,r4@7,r13@8,r5@9,r14@10,r11@22}\\
\hline
  B2 &
  \tabincell{c}{r1@1,r10@2,r2@3,r11@4,r3@5,r15@6,r4@7,r13@8,r5@9,\\
  r14@10,r4@15,r14@16,r1@17,r7@18,r15@19,r11@22} &
  \tabincell{l}{r1@1,r10@2,r2@3,r11@4,r3@5,r15@6,r4@7,r13@8,r5@9,
    \\r6@12,r14@13,r4@15,r1@17,r7@18,r15@19,r11@22} \\
\hline
  B3 & \tabincell{c}{r1@1,r10@2,r2@3,r11@4,r3@5,r15@6,r4@7,r13@8,r5@9,
    \\r6@12,r14@13,r4@15,r1@17,r7@18,r15@19,r11@22} &
  \tabincell{l}{r10@2,r2@3,r3@5,r13@8,r5@9,r6@12,\\r14@13,r4@15,r14@16,r1@17,r7@18,r15@19} \\
  \hline                       
  B4 & \tabincell{l} {r1@1,r10@2,r2@3,r11@4,r3@5,r15@6,r4@7,r13@8,
    \\r5@9,r14@10,r6@12,r14@13,r4@15,r1@17,r7@18,r15@19,r11@22} &
  \tabincell{l}{r1@1,r10@2,r2@3,r3@5,r13@8,r14@10, 
    \\r6@12,r14@13,r1@17,r7@18,r4@21,r11@22,r5@23,r15@24} \\
\hline
  B5 & \tabincell{l}{r1@1,r10@2,r2@3,r3@5,r13@8,r14@10, 
    \\r6@12,r14@13,r1@17,r7@18,r4@21,r11@22,r5@23,r15@24} &
  \tabincell{l}{r1@1,r2@3,r3@5,r13@8,r14@10,r14@13,
\\r1@17,r7@18,r4@21,r5@23,r15@24,r6@26,r10@27,r8@28,r11@29}
  \\
  \hline  
\end{tabular}
\\\\iter.2
\\ IN(b) = $\bigcup_{p\in pred(b)} OUT(p) $\\ 
OUT(b)=Gen(b) $\cup$ (IN(b) $\cap$ PRSV(b)) = Gen(b) \\\\
\begin{tabular}{| l | c | l |} 
 \hline
 BB & IN & OUT \\
  \hline                       
  B0 & $\emptyset$ &{r1@1,r10@2,r2@3,r11@4,r3@5 }\\
\hline
  B1 &
  \tabincell{c}{r1@1,r10@2,r2@3,r11@4,r3@5,r13@8,r14@10,r6@12,r14@13\\r1@17,r7@18,
    r4@21,r11@22,r5@23,r15@24} & \tabincell{l}
  {r1@1,r10@2,r2@3,r11@4,r3@5,r15@6,r4@7,r13@8,r5@9,\\r14@10,r1@17,r7@18,r11@22}\\
\hline
  B2 &
  \tabincell{c}{r1@1,r10@2,r2@3,r11@4,r3@5,r15@6,r4@7,r13@8,r5@9,\\
  r14@10,r6@12,r14@13,r4@15,r14@16,r1@17,r7@18,r15@19,r11@22} &
  \tabincell{l}{r1@1,r10@2,r2@3,r11@4,r3@5,r15@6,r4@7,r13@8,r5@9,
    \\r6@12,r14@13,r4@15,r1@17,r7@18,r15@19,r11@22} \\
\hline
  B3 & \tabincell{c}{r1@1,r10@2,r2@3,r11@4,r3@5,r15@6,r4@7,r13@8,r5@9,
    \\r6@12,r14@13,r4@15,r1@17,r7@18,r15@19,r11@22} &
  \tabincell{l}{r10@2,r2@3,r3@5,r13@8,r5@9,r6@12,\\
    r14@13,r4@15,r14@16,r1@17,r7@18,r15@19} \\
  \hline                       
  B4 & \tabincell{l} {r1@1,r10@2,r2@3,r11@4,r3@5,r15@6,r4@7,r13@8,
    \\r5@9,r14@10,r6@12,r14@13,r4@15,r1@17,r7@18,r15@19,r11@22} &
  \tabincell{l}{r1@1,r10@2,r2@3,r3@5,r13@8,r14@10, 
    \\r6@12,r14@13,r1@17,r7@18,r4@21,r11@22,r5@23,r15@24} \\
\hline
  B5 & \tabincell{l}{r1@1,r10@2,r2@3,r3@5,r13@8,r14@10, 
    \\r6@12,r14@13,r1@17,r7@18,r4@21,r11@22,r5@23,r15@24} &
  \tabincell{l}{r1@1,r2@3,r3@5,r13@8,r14@10,r14@13,
\\r1@17,r7@18,r4@21,r5@23,r15@24,r6@26,r10@27,r8@28,r11@29}
  \\
  \hline  
\end{tabular}
\\\\iter.3
\\No change\\
(c)\\
B0 $r1\Rightarrow r1@1,r2\Rightarrow r2@3$\\
B1 $r3\Rightarrow r3@5,r10\Rightarrow r10@2,r11\Rightarrow r11@4 or
r11@22,r4\Rightarrow r4@7,r13\Rightarrow r13@8,r10\Rightarrow
r10@2,r5\Rightarrow r5@9, r14\Rightarrow r14@10$\\
B2 $r11\Rightarrow r11@4 or r11@22,r14\Rightarrow r14@13,r6\Rightarrow
r6@12,r14\Rightarrow r14@13$ \\
B3 $r10\Rightarrow r10@2,r11\Rightarrow r11@22 or r11@4,r4 \Rightarrow
r4@15,r15\Rightarrow r15@19 or r15@6,r1\Rightarrow r1@17,r7\Rightarrow
 r7@18$\\
B4 $r10\Rightarrow r10@2,r11\Rightarrow r11@22 or r11@4,r4\Rightarrow 
r4@21,r13\Rightarrow r13@8,r10\Rightarrow r10@2,r5\Rightarrow
r5@23,r15\Rightarrow r15@24$\\
B5 $r11\Rightarrow r11@22,r14\Rightarrow r14@13, r6\Rightarrow
r6@26,r10\Rightarrow r10@27,r14\Rightarrow r14@13,r8\Rightarrow r8@28$
}
%\begin{landscape}
%\end{landscape}
\\\\\\
2.Answer the following questions related to liveness analysis analysis
for the CFG
\\
(a) Give the GEN and PRSV sets for each basic block
{\small
\\
$U=\{r1,r2,r3,r4,r5,r6,r7,r8,r10,r11,r13,r14,r15 \}$
\\
\\
\begin{tabular}{| l | c | l |} 
 \hline
 BB & GEN & PRSV\\
  \hline                       
  B5 & {r11,r14 } & \tabincell{l}
  {r1,r2,r3,r4,r5,r7,r13,r14,r15} \\
\hline
  B4 & {r10,r11,r13} & \tabincell{l} {r1,r2,r3,r6,r7,r8,r10,r13,r14}\\
\hline
  B3 & {r10,r11,r15} &
  \tabincell{l}{r2,r3,r5,r6,r8,r10,r13} \\
\hline
  B2 & {r11,r14} &
  \tabincell{l}{r1,r2,r3,r4,r5,r7,r8,r10,r11,r13,r15} \\
  \hline                       
  B1 & {r3,r10,r11} &  \tabincell{l}{r1,r2,r3,r6,r7,r8,r10,r11} \\
\hline
  B0 & $\emptyset$ &
  \tabincell{l}{r4,r5,r6,r7,r8,r13,r14,r15} \\
  \hline  
\end{tabular}
\\\\\\
(b)Liveness is a backwards problem, That is because to verify a node
is liveness or not we , so we need to take postOrder to analysis it.
\\\\
(c)
\\Init.(iter.0)\\
OUT(b) = $\emptyset$,
IN(b)=Gen(b) $\cup$ (OUT(b) $\cap$ PRSV(b)) = Gen(b) \\\\
\begin{tabular}{| l | c | l |} 
 \hline
 BB & OUT & IN\\
  \hline                       
  B5 & $\emptyset$ &{r11,r14}\\
\hline
  B4 & $\emptyset$ &{r10,r11,r13} \\
\hline
  B3 & $\emptyset$& {r10,r11,r15}  \\
\hline
  B2 & $\emptyset$& {r11,r14}  \\
  \hline                       
  B1 & $\emptyset$& {r3,r19,r11} \\
\hline
  B0 & $\emptyset$& $\emptyset$ \\
  \hline  
\end{tabular}
\\\\iter.1
\\ OUT(b) = $\bigcup_{s\in succs(b)} IN(s) $\\ 
IN(b)=Gen(b) $\cup$ (OUT(b) $\cap$ PRSV(b)) = Gen(b) \\\\
\begin{tabular}{| l | c | l |} 
 \hline
 BB & OUT & IN \\
  \hline                       
  B5 & $\emptyset$ &{r11,r14}\\
\hline
  B4 & {r3,r11,r14,r10} & \tabincell{l} {r3,r10,r11,r13,r14}\\
\hline
  B3 &
  \tabincell{c}{r11,r14} &
  \tabincell{l}{r10,r11,r15} \\
\hline
  B2 & \tabincell{c}{r3,r10,r11,r13,r14,r15} &
  \tabincell{l}{r3,r10,r11,r13,r14,r15} \\
  \hline                       
  B1 & \tabincell{l} {r3,r10,r11,r13,r14,r15} &
  \tabincell{l}{r3,r10,r11} \\
\hline
  B0 & \tabincell{l}{r3,r10,r11} &
  \tabincell{l}{$\emptyset$}
  \\
  \hline  
\end{tabular}
\\\\iter.2
\\ OUT(b) = $\bigcup_{s\in succs(b)} IN(s) $\\ 
IN(b)=Gen(b) $\cup$ (OUT(b) $\cap$ PRSV(b)) = Gen(b) \\\\
\begin{tabular}{| l | c | l |} 
 \hline
 BB & OUT & IN \\
  \hline                       
  B5 & $\emptyset$ &{r11,r14}\\
\hline
  B4 & {r3,r11,r14,r10} & \tabincell{l} {r3,r10,r11,r13,r14}\\
\hline
  B3 &
  \tabincell{c}{r3,r10,r11,r13,r14,r15} &
  \tabincell{l}{r3,r10,r11,r13,r14,r15} \\
\hline
  B2 & \tabincell{c}{r3,r10,r11,r13,r14,r15} &
  \tabincell{l}{r3,r10,r11,r13,r14,r15} \\
  \hline                       
  B1 & \tabincell{l} {r3,r10,r11,r13,r14,r15} &
  \tabincell{l}{r3,r10,r11} \\
\hline
  B0 & \tabincell{l}{r3,r10,r11} &
  \tabincell{l}{$\emptyset$}
  \\
  \hline  
\end{tabular}
\\\\iter.3
\\ OUT(b) = $\bigcup_{s\in succs(b)} IN(s) $\\ 
IN(b)=Gen(b) $\cup$ (OUT(b) $\cap$ PRSV(b)) = Gen(b) \\\\
No change
\\\\
3 living analysis
\\(a)pointer reference.
\\if we could know which variable the pointer reference, we can just
use this variable instead, however, sometimes we could not verify
until run-time, so we need to find all possible variables, the simple
ideas is to think it points to all variables.
\\
Assume pointer on the right, GEN need to insert ,PRSV no change,
Assume pointer on the left, GEN no change ,PRSV becomes empty set,
\\(b)procedure calls \\
Same situations for procedure calls.
functions can do anything – redefine variables, use variables
So DEF(foo()) is $\emptyset$ and USE(foo()) is $\bigcup$ 
\\\\4.sinking transformation.
\\(a)\\
Sinking is a transformation that move an instruction to its succsor
when its result is first needed. So we need make sure the operands of
the instruction could not be changed during its node to sinking point,
also we need to find where the lvalue of the instruction is first
used, then copy the instruction here and eliment the former one.
\\
Forward, allpath problem.
\\(b)
available set. if found the instruction in this set, we need to delete
,if found the operand in this set we insert a new instruction and
delete the sets.
\end{document}
