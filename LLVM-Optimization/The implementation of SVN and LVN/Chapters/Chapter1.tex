% Chapter 1

%\titleformat{\chapter}{\centering\Huge\bfseries}{Part \,\thechapter\}{1em}{}

%\chapter{Introduction} % Main chapter title

\label{Chapter1} % For referencing the chapter elsewhere, use \ref{Chapter1} 

\lhead{ \emph{Value Numbering Analysis}} % This is for the header on each page - perhaps a shortened title

%----------------------------------------------------------------------------------------

\section{Introduction}

This class uses global value numbering,GVN, to eliminate fully
redundant instructions.  It also performs simple dead load
elimination together.\\
To implent global value numbering, GVN create the value table class
to  hold the mapping relation between values and value numbers, it also uses
a structure LeaderTable which mapping from value numbers to lists of Value*'s which
have same value number. 
\\
Basically, GVN does the valuing number work based on a
domainator tree.It uses a dependency method to get the domaint tree 
,  and then use a DFS iteration go through the whole dominator tree.
This happens in the function "iterateOnFunction".Then it is
instruction processing.
 When meet an instruction, first check it could be
emliminated  or not(it is not the valuing job, but only for
convient, like merge block, simplicathin instruction),
then put the instruction into VN(value nuber table),
aslo add to LeaderTable, this will generate two numbers.
According the two numbers, it will compare the current
LeaderTableEntry with the other basic block numbers in the 
dominator tree and dominates it.If found one, means, 
current one could be eliminated. Since a function has one
dominator tree, we only do clean job once.

\section{Implementation}
Based on the analysis of GVN, we could find that, SVN and LVN could
be easily implented by offering the class a special block and limited 
the checing range to this special blocks. For GVN, it is a dominator
tree,For SVN it is an EBB, and For LVN , it is only one single basic
block. 

\subsection{SVN} 
UNlike  the GVN, I did not care about the load emiliatin so much, what
I do is use an EBBForest structure instead of domainator tree, iterating
EBBTree sepeatly, and insert value to vn and LeaderTable.When
doing findleader job, I only compare the basic block with those blocks
 in LeaderTable and could extend it during the EBBTree.  

\subsubsection{LVN} 
Same idea about SVN Implementation, I processed one block one time. 
 
\subsection{Command-line}
There are two mechinism of argument passing in LLVM, one is use the
parament directly, like "Release+Asserts/bin/opt -svn test.bc", 
in this way, I  need to put this command to std-compiler-opts,and
solve some problems about the dependency relationship. So far, I only
find the flow of a argument, there are still some work on how let a
svn class use gvn class well.\\ 
There is another  way,easy way, just use it as a condition variable of GVN
,like "Release+Asserts/bin/opt -gvn --enable-lvn=true test.bc" ,
 then add “static cl::opt<bool> EnableLVN("enable-lvn",cl::init(false), cl::Hidden);”
to front of the file, then we could use EnableLVN as a global bool
type variable in this file.
\subsection{EBB} 
I designed a EBB pass in this way:\\
First, construct an EBB tree class to implent an Extend basic block,
it has a root to save the entry block, it also has a vector structure
to save all block. I tried to create a new map structure to 
map a basic block ant its ancestors, but because of the time,
I did not finish yet.This class also provides some functions like
verify the ancestor. \\
Secondly, I counstruct an EBBForest class to save all ebbs in a
function. Considering there would be multiple functions in file,
I create an map structure to connect the function name and ebb forest.
\\
The main function Implemented in "runOnFunction", I also provide a
function to save all ebbs in a file.

\
%----------------------------------------------------------------------------------------


\section{Result anlysis}
The result look like below:
\begin{center}
\begin{tabular}{|c|r|c|c|}
\hline
Delete NO & GVN  & SVN  & LVN \\
\hline
Bubblesort & 7 & 7 & 7  \\
\hline
FloatMM & 5 & 6 & 6  \\
\hline
IntMM & 6 & 6 &6 \\
\hline
Oscar & 7 & 7 & 6 \\
\hline
Perm & 3 & 3 & 2 \\
\hline
Puzzle & 23 & 17 & 27 \\
\hline
Queens & 3 & 3 & 0 \\
\hline
Quicksort & 14 & 13 & 12\\
\hline
RealMM & 6 & 6 & 6 \\
\hline
Towers & 5 & 5 & 59  \\
\hline
Treesort & 13 & 13 & 13\\
\hline
\end{tabular}
\end{center}
\begin{center}
\begin{tabular}{|c|r|c|c|}
\hline
Time & GVN  & SVN  & LVN\\ 
\hline
Bubblesort & 0.0002 & 0.0002 & 0.0002  \\
\hline
FloatMM & 0.0002  & 0.0002 & 0.0002  \\
\hline
IntMM & 0.0002 & 0.0002 & 0.0001 \\
\hline
Oscar & 0.0004 & 0.0004 & 0.0003 \\
\hline
Perm & 0.002 & 0.0003 & 0.0002 \\
\hline
Puzzle & 0.001 & 0.0009 & 0.001\\
\hline
Queens & 0.0002 & 0.0002 & 0.0002  \\
\hline
Quicksort & 0.0002 & 0.0002 & 0.0002\\
\hline
RealMM & 0.0001 & 0.0001 &  0.0001\\
\hline
Towers & 0.0004 & 0.0004 & 0.0005  \\
\hline
Treesort & 0.0003 & 0.0003 & 0.0003\\
\hline
\end{tabular}
\end{center}
\begin{center}
\begin{tabular}{|c|r|c|c|}
\hline
Compile T & GVN  & SVN  & LVN\\ 
\hline
Bubblesort & 0.004 & 0.0039 & 0.004  \\
\hline
FloatMM & 0.0048 & 0.0047 & 0.00473  \\
\hline
IntMM & 0.0054 & 0.005 & 0.0046 \\
\hline
Oscar & 0.0084 & 0.0084 & 0.0087 \\
\hline
Perm & 0.0031 & 0.0032 & 0.003 \\
\hline
Puzzle & 0.0286 & 0.0279 & 0.0276\\
\hline
Queens & 0.0051 & 0.005 & 0.005  \\
\hline
Quicksort & 0.0043 & 0.0043 & 0.0038\\
\hline
RealMM & 0.0044 & 0.0045 & 0.0045 \\
\hline
Towers & 0.006 & 0.0058 & 0.0060  \\
\hline
Treesort & 0.0052 & 0.0051 & 0.0051\\
\hline
\end{tabular}
\end{center}
\begin{center}
\begin{tabular}{|c|r|c|c|}
\hline
Execute & GVN  & SVN  & LVN \\
\hline
Bubblesort & 0.005 & 0.005 & 0.005  \\
\hline
FloatMM & 0.001 & 0.004 & 0.006  \\
\hline
IntMM & 0.005  & 0.005 & 0.004 \\
\hline
Oscar & 0.009 & 0.009 & 0.011 \\
\hline
Perm & 0.0002 & 0.004 & 0.003 \\
\hline
Puzzle & 0.03 & 0.028  & 0.025\\
\hline
Queens & 0.006 & 0.004 & 0.004  \\
\hline
Quicksort & 0.003 & 0.004 & 0.003\\
\hline
RealMM & 0.003 & 0.004 & 0.004 \\
\hline
Towers & 0.004 & 0.006 & 0.005  \\
\hline
Treesort & 0.006 & 0.004  & 0.003\\
\hline
\end{tabular}
\end{center}
%----------------------------------------------------------------------------------------


\section{Conclution}
According to the data below, we could find that:\\
1. Comiler time takes a big proportion of the total time, nealy equal
to execute time, even bigger sometimes.\\
2. From the exectute time, we find that gvn $<$ svn$<$lvn, which means a
good optimization does improve execute effiency of program \\
3. Each optimization time is less than 10 percents of total compiler
time, so optimization is necessary.\\ 
4. There is a relationship between code numbers and execute time,
usually less code less time.\\
5. A good opitimization improves effiency indeed, but it will improve
the compile time at the same time. So a good algorithm is important
even for opitimization Implementation.
  

%----------------------------------------------------------------------------------------


\begin{flushright}
%Guide written by ---\\
%Sunil Patel: \href{http://www.sunilpatel.co.uk}{www.sunilpatel.co.uk}
\end{flushright}
