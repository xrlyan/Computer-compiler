% Chapter 1

%\titleformat{\chapter}{\centering\Huge\bfseries}{Part \,\thechapter\}{1em}{}

%\chapter{Introduction} % Main chapter title

\label{Chapter1} % For referencing the chapter elsewhere, use \ref{Chapter1} 

\lhead{ \emph{Self Spatial Reuse}} % This is for the header on each page - perhaps a shortened title

%----------------------------------------------------------------------------------------

\section{Introduction}

1. compile  the source file

clang -O1 -emit-llvm -c loops.c 

2. run tests   

opt -analyze -ds --ssr=true --cls=64 loops.bc

\section{Implementation}
\subsection{Basic Algorithm}
A memory reference in a loop nest has self-spatial reuse if the
distance (stride) between two consecutive accesses by this reference
in the innermost loop is smaller than the cache line size.

I only think about the self spatial reuse at innermost level ,
 for example I J K order, I only think about
 A[I][J][K], A[I][K][K], A[K][K][K].
If no K at all, I take it as a self-temp and pass over the dectect.
For example,
\begin{lstlisting}
  for (i = 1; i < N; i++)
      for (j = 1; j < N; j++)
          for (k = 1; j < N; j++)
          {
            a[k][j][k] = a[i][j][j] + a[i][k][k]+a[i][i][k];
          }
\end{lstlisting}
the stride of a[i][j][k] is 1, the stride of a[i][k2][k3] is
upperbound of L3 * stride of k2 + stride of k3, k2 means k in level
2 and k3 means k in level 3. Same to a[k1][j][k3]. But when meets
a[i][j][j], I will take it as a self-temp. 

At last we will make sure the stride count will be lower the
cachelinecount, which means two consecutive value store in  a same
cache line.

\subsection{Loop Nest Number} 
Find the outermost loop (no parent) of each loopnest, save it to a
currentLoop, then , find another outermost, check if it is same as the
currentLoop, if same, do nothing, not same, let loopNest++, and make
currentLoop = current outermost loop.

\subsection{Get the variable type} 
Need to know the value type of the array, because different type comes
with different layout size. Int would be 4, and doulbe is 8, which
means a cacheline with size 64 can hold 16 integers but 8 doulbe variable. 
This is one thing we need to notice, if we get a pointer value, we
need to find the element it points to.

\subsection{Get the subscripts} 
   Maily using the Pair struct offered by this DA file. For example 
\begin{lstlisting}
  for (i = 1; i < N; i++)
      for (j = 1; j < N; j++)
      {
        a[i][j] = a[i][2*j] + 10;
      }

\end{lstlisting}
we will get [i] of a[i][j] from Pair[1], and could get the i of a[i]
from Loop 1, also we could get the coeffient 1, which we could use it
as the stride Same idea with [j], we know its coeff is 1 and loop
level is 2, we need its stride is lower or equal to cachelinecount 

\subsection{Get the upperbound}
Upperbound is necessary for a[i][k][k] and a[k][j][k], this kind of
situations. We also need to get all level upperbound in advance.

\subsection{MIV} 
For array A[i][j+k][i], we need do a loop for all j and k in j+k, and
check if it is same as k, if same, we need to use upperbound to
caculate the stride count. 


%----------------------------------------------------------------------------------------
\section{The difficulties}
For A[i][j][j], there is a situation that, cacheline is very big, and
upperbound of lever 2 and 3 is small, then although level 3 has a
self-temp, level 2 still could have a self-reuse, which means
a[i][j][j] and a[i][j+1][j+1] could be in a same cache line.
Considering its big possibily accroding to dimmentions, I did not
handle this case this time. 

%----------------------------------------------------------------------------------------

\newpage
\section{Results}
\begin{lstlisting}
test1

test2

Loop Nest : 1
  %1 = load i32* %arrayidx, align 8, !tbaa !1
level : 1
stride : 2

  store i32 %1, i32* %arrayidx2, align 4, !tbaa !1
level : 1
stride : 1

test3

Loop Nest : 1
  %1 = load i32* %arrayidx, align 8, !tbaa !1
level : 1
stride : 4

  store i32 %1, i32* %arrayidx2, align 4, !tbaa !1
level : 1
stride : 2

test4

Loop Nest : 1
  %2 = load i32* %arrayidx, align 4, !tbaa !1
level : 1
stride : 8

  %4 = load i32* %arrayidx3, align 16, !tbaa !1
level : 1
stride : 12

test5

test6

Loop Nest : 1
  %3 = load i32* %arrayidx5, align 4, !tbaa !1
level : 2
stride : 1

  store i32 %add10, i32* %arrayidx14, align 4, !tbaa !1
level : 2
stride : 1

test7

Loop Nest : 1
  %4 = load i32* %arrayidx6, align 8, !tbaa !1
level : 2
stride : 2

  %6 = load i32* %arrayidx11, align 4, !tbaa !1
level : 2
stride : 3

  store i32 %add12, i32* %arrayidx17, align 4, !tbaa !1
level : 2
stride : 1

test8

Loop Nest : 1
  store i32 %add, i32* %arrayidx13, align 4, !tbaa !1
level : 2
stride : 3

test9

Loop Nest : 1
  %0 = load double* %arrayidx10, align 8, !tbaa !1
level : 3
stride : 1

  %3 = load double* %arrayidx17, align 8, !tbaa !1
level : 3
stride : 2

  store double %add18, double* %arrayidx25, align 8, !tbaa !1
level : 3
stride : 1

test10

Loop Nest : 1
  %4 = load double* %arrayidx17, align 8, !tbaa !1
level : 3
stride : 4

  store double %add18, double* %arrayidx24, align 16, !tbaa !1
level : 3
stride : 2

test11

test12

Loop Nest : 1
  %1 = load double* %arrayidx10, align 8, !tbaa !1
level : 3
stride : 5
\end{lstlisting}

%----------------------------------------------------------------------------------------


\begin{flushright}
%Guide written by ---\\
%Sunil Patel: \href{http://www.sunilpatel.co.uk}{www.sunilpatel.co.uk}
\end{flushright}
